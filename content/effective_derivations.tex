\chapter{Effective Derivations}
In this section we prove that several sets of constraints corresponding to natural combinatorial optimization problems are effective.
We first note that the very simplest sets of constraints are effective.
\begin{lemma}\label{lem:grobnereffective}
Let $\cP$ be a Gr\o bner basis. Then $\cP$ is $1$-effective in the PC proof system.
\end{lemma}
\begin{proof}
Recall that if $\cP$ is a Gr\o bner basis, then multivariate polynomial division by $\cP$ is well-defined, and in fact there is a unique 
remainder. Let $r \in \gen{\cP}$ be of degree $d$, and consider the division of $r$ by $\cP$. Because $r \in \gen{\cP}$, this division must have
remainder zero. If we enumerate the polynomials that are produced in the course of this division $r = r_0, r_1, \dots, r_N = 0$, then 
$r_i = r_{i+1} + q_{i+1}p_{i+1}$, where $p_{i+1} \in \cP$ and $\deg r_{i+1} \leq \deg r_i$. Combining all these sums into one, we get
$r = \sum_i q_i p_i$, which is a derivation of degree $d$. 
\end{proof}
\prettyref{lem:grobnereffective} is trivial to prove, but there are several important combinatorial optimization problems that fall 
under its umbrella.
\begin{corollary}
\textsc{CSP} is $1$-effective.
\end{corollary}
\begin{proof}
Recall that \textsc{CSP} corresponds to the set of constraints $\cP = \{x_i^2 - x_i | i \in [n]\}$. We prove this is a Gr\o bner basis. 
Let $p \in \gen{\cP}$. If $p$ is not multilinear, we can divide $p$ by elements of $\cP$ until we have a multilinear remainder $r$. Because 
$p \in \gen{\cP}$ and each element of $\cP$ is zero on the hypercube $\{0, 1\}^n$, $r$ must also be zero on the hypercube. But the multilinear
polynomials form a basis for functions on the hypercube, so if $r$ is a multilinear polynomial which is zero, then it must be the zero polynomial. 
\end{proof}
\begin{corollary}
\textsc{CLIQUE} is $1$-effective.
\end{corollary}
\begin{proof}
Recall that \textsc{CLIQUE}$(V,E)$ corresponds to the set of constraints $\cP = \{x_i^2 - x_i | i \in [|V|]\} \cup \{ x_ix_j | (i,j) \notin E\}$. 
We prove this is a Gr\o bner basis. Let $p \in \gen{\cP}$. Just as above, if $p$ is not multilinear, we can divide it until we have a multilinear
remainder $r_1$. Now by dividing $r_1$ by the polynomials in the second part of $\cP$, we can remove all monomials containing $x_ix_j$ where $(i,j) \notin E$
to get $r_2$. Thus $r_2$ contains only monomials which are cliques of varying sizes in the graph $(V,E)$. Let $C$ be the smallest clique with a
nonzero coefficient $\alpha_C$ in $r_2$. Let $\chi_C$ be the characteristic vector of $C$, i.e. $(\chi_C)_i = 1$ if $i \in C$, and $(\chi_C)_i = 0$ otherwise.
Then $r_2(\chi_C) = \alpha_C$. But $p(\chi_C) = 0$ for every $p \in \cP$, and $r_2 \in \gen{\cP}$. Thus $\alpha_C = 0$, a contradiction, and so $r_2$ is the 
zero polynomial.
\end{proof}

However, not every problem falls so neatly into this classification. There are many natural problems whose solution spaces have a small set of generating
polynomials which are not Gr\o bner bases, and indeed their Gr\o bner bases can be exponentially large and exceedingly difficult to compute. Despite this,
these problems can still admit effective derivations, as we prove here.

\section{More Complicated Solution Spaces}
In this section, we will prove that the \textsc{Matching} and \textsc{Balanced-CSP} problems have effective solution spaces. Both of the solution spaces have
two properties that are important for our proofs. Firstly, they are very symmetric: one can permute the vertices of a matching however one wishes and still have a matching. Similarly, permuting the names of variables does not change the balance of an assignment. 
Secondly, they are inductive: deleting an edge from a matching results in a matching on a smaller graph. 
Similarly, removing a pair of variables with different values gives you an assignment, although it could be differently balanced.
The proofs described in this section should be adaptable to any problem whose solution space has similar properties.

Our proofs are by induction, and they proceed in two steps. 
First, we show that any perfectly symmetric polynomial $p$ of degree $d$ can be derived from a constant polynomial in degree $d$.
This step is easy and straightforward. 
The second, and much harder, step is to show that given a $p$ which is constant on $V(\cP)$ and any permutation $\sigma$, the polynomial $p - \sigma p$ can be derived in low degree. 
This second step is accomplished with an inductive argument that shows how to use $\cP$ to strip off two of the variables of the solutions. 
The group action that determines the definition of "`perfectly symmetric"' and $\sigma p$ has not been specified yet, but it will be the natural permutation of the underlying solution space.

%Intuitively speaking, this means that their solution spaces are simple and easily explained. Any polynomial fact over the solution space
%can be proven using a proof of only roughly the same complexity of the fact.  