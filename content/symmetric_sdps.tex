\chapter{Optimal Symmetric SDP Formulations and a Lower Bound}\label{cha:symmetric_sdps}
The main result of this section is to show that when the solution space of a polynomial formulation for a combinatorial optimization problem satsfies certain symmetry properties, then the Theta Body SDP relaxation (see \prettyref{sec:polyforms}) achieves the best approximation among all symmetric SDPs of a comparable size. This is proven using an old theorem of Yannakakis that has been used time and time again to find optimal symmetric LP and SDP relaxations \cite{TODO}. 

We combine this result with some of our results in \prettyref{cha:effective_derivations} to prove that the Sum-of-Squares SDP relaxation (see \prettyref{sec:polyforms} again) performs no worse than the Theta Body relaxation, thus showing that the SOS SDP is optimal for problems including \textsc{Matching}, \textsc{TSP}, and \textsc{Balanced CSP}. Furthermore, this allows us to translate lower bounds against the SOS SDP to lower bounds against any symmetric SDP formulation. We apply this to the \textsc{Matching} problem using the lower bound of Grigoriev \cite{TODO}.

\section{Theta Body Optimality}
Recall that a $S_m$-symmetric combinatorial optimization problem $\cM = (\cS, \cF)$ has a symmetric polynomial formulation if two conditions hold. First, there is a polynomial optimization problem $(\cP, \cQ, \cO, \phi)$ on $n$ variables such that solving the associated problem solves $\cM$ as well. Second, there is an action of $S_m$ on the coordinates $[n]$ (extending to an action on $\R^n$) that is compatible with the action on $\cS$: $\sigma\phi(s) = \phi(\sigma s)$. This action extends to an action on polynomials simply by $\sigma x_i = x_{\sigma i}$ and extending it linearly and multiplicatively. 
\begin{definition}
We say that the symmetric polynomial formulation is \emph{$(k_1,k_2)$-block transitive} if, for each $U \subseteq [m]$ of size at most $k_1$, there exists a $V \subseteq [n]$ of size at most $k_2$ such that $A([m] \setminus U)$ acts transitively on each $S_{V,c} = \{x \in \R^n: x \in V(\cP), x|_V = c\}$.
\end{definition}
\begin{example}\label{ex:matching-blocktransitive}
\textsc{Matching} is $\left(k,\binom{k}{2}\right)$-block transitive for each $k < n/2$. Recall the constraints of the polynomial formulation for \textsc{Matching} on $\binom{n}{2}$ variables:
\begin{align*}
&\cP_n = \matching \\
&\phi_n(M) = \chi_M \text{ such that } (\chi_M)_{ij} = 1 \text{ if $(i,j) \in M$ and $0$ otherwise.}
\end{align*}
Then for a subset $I \subseteq [n]$ with $|I| < n/2$, we set $J = E(I,I)$, the set of edges that lie entirely in $I$. Let $M_1$ and $M_2$ be two matchings that agree on $J$. We define a permutation $\sigma$ as follows: Set $\sigma$ to fix $I$. Because $M_1$ and $M_2$ are perfect matchings, they must have the same number of edges in both $E(I, \overline{I})$ and $E(\overline{I},\overline{I})$. For a vertex $v \in \overline{I}$, if $M_1(v) \in I$, then we set $\sigma(v) = M_2(M_1(v))$. Otherwise, we set $\sigma$ to be an arbitrary bijection between the edges of $M_1$ in $E(\overline{I},\overline{I})$ and the edges of $M_2$ in $E(\overline{I},\overline{I})$. Clearly $\sigma \in S([n]\setminus I)$ and $\sigma(\chi_{M_1}) = \chi_{M_2}$. If $\sigma$ is even, we are done. Otherwise, since $|I| < n/2$, there is an edge $(u,v) \in M_2 \cap E(\overline{I},\overline{I})$. Then $(u,v) \cdot \sigma$ is an even permutation which still fixes $J$ and maps $\chi_{M_1}$ to $\chi_{M_2}$.
\end{example}

\begin{example}\label{ex:bcsp-blocktransitive}
\textsc{Balanced CSP} is $(k,k)$-block transitive for every $k \leq n-3$. Recall the constraints for the polynomial formulation for \textsc{Balanced CSP} on $n$ variables with balance $c$:
\begin{align*}
&\cP(n,c) = \bcsp{n}{c} \\
&\phi_n(A) = \chi_A \text{ such that } (\chi_A)_i = 1 \text{ if $A(i) = 1$ and $0$ otherwise.}
\end{align*}
Then for a subset $I \subseteq [n]$, we set $J = I$. Let two assignments $A_1$ and $A_2$ that agree on $J$, and define a permutation $\sigma$ as follows: $\chi_{A_1}$ and $\chi_{A_2}$ have the same number of indices which are zero, and indices which are one. Let $\sigma$ be any pair of bijections between the indices which are one in $\chi_{A_1}$ and the indices which are one in $\chi_{A_2}$, and likewise for the indices which are zero. Furthermore, since $\chi_{A_1}$ and $\chi_{A_2}$ agree on $J$, we can choose $\sigma$ to be a pair of bijections which are the identity on $J$, so $\sigma \in S([n] \setminus J)$. Clearly $\sigma(\chi_{A_1}) = \chi_{A_2}$. Finally, if $\sigma$ is not already even, since $|I| \leq n-3$, there are two indices $\ell_1$ and $\ell_2$ outside of $J$ such that $A_2(\ell_1) = A_2(\ell_2)$. Then $(\ell_1, \ell_2) \cdot \sigma$ is even and still fixes $J$ and maps $\chi_{A_1}$ to $\chi_{A_2}$.
\end{example}

The point of block-transitivity is that it allows us determine that certain functions can be represented with low-degree polynomials. Going from arbitrary functions to low-degree polynomials is crucial to showing optimality for the Theta Body.  
\begin{lemma}
Let $(\cP, \cO, \phi)$ be a $A_m$-symmetric, boolean, $(k_1,k_2)$-block transitive polynomial formulation and $h: V(\cP) \rightarrow \R$ be a function with group action $\sigma h(x) = h(\sigma^{-1} s)$. If there is a set $I$ of size $|I| \leq k_1$ such that $h$ is stabilized by $A([m] \setminus I)$, then there is a polynomial $h'(x)$ such that $h'(\phi(s)) = h(\phi(s))$ and the degree of $h'$ is at most $k_2$. 
\end{lemma}
\begin{proof}
For any $\sigma \in A([n] \setminus I)$ and $\alpha \in V(\cP)$, we know $h(\alpha) = \sigma h(\alpha) = h(\sigma^{-1} \alpha)$. By block-transitivity, there exists a set $J$ of size $|J| \leq k_2$ such that $A([m]\setminus I)$ acts transitively on elements of $V(\cP)$ which agree on $J$. Thus if $\alpha,\beta \in V(\cP)$ such that $\alpha|_J = \beta|_J$, $h(\alpha) = h(\beta)$. Thus $h$ depends only on the coordinates $J$, and since the polynomial formulation is boolean, any such function can be expressed as a degree $|J|$ polynomial. 
\end{proof} 

Before we state our main theorme, we recall that when $\cM$ has a polynomial formulation where $\cQ = \emptyset$, we can define the $d$th Theta Body relaxation with objective $o(x)$ as 
\[\begin{tabular}{ll} $\min c$ & subject to \\ $c - o(x)$ is $d$-SOS modulo $\gen{\cP}$.\end{tabular}\] 
\begin{theorem}\label{thm:symmetric-main}
Let $\cM = (\cS, \cF)$ have a $S_m$-symmetric $(k_1,k_2)$-block transitive boolean polynomial formulation $(\cP, \cO, \phi)$ on $n$ variables. Then if $\cM$ has any $(C,S)$-approximate, $S_m$-symmetric SDP relaxation of size $r < \sqrt{\binom{n}{k_1}}$ the $k_2$th Theta Body relaxation is a $(C,S)$-approximate relaxation as well.
\end{theorem}
Before we prove the main theorem, we need two lemmas. One has to do with obtaining sum-of-squares representations for the objective functions given a small SDP formulation:

\begin{lemma}\label{lem:sdptosos}
If $\cM = (\cS, \cF)$ has a $(S,C)$-approximate SDP formulation of size at most $k$, then there exist a family of $\binom{k+1}{2}$ functions $\cH$ from $\cS$ into $\R$ such that for every $f \in \cF$, with $\max f \leq S(f)$,
\[C(f) - f = \sum_i g_i^2\]
where each $g_i \in \gen{\cH}$. Furthermore, if the SDP formulation is $G$-coordinate-symmetric for some group $G$, then $\cH$ is $G$-invariant under the action $\sigma h(s) = h(\sigma^{-1}s)$.
\end{lemma} 
\begin{proof}
Consider the slack matrix for $\cM$: $M(s,f) = C(f) - f(s)$. By \prettyref{thm:yannakakis}, if there exists an SDP formulation for $\cM$ of size $k_1$, then there are $k_1 \times k_1$ PSD matrices $X^s$ and $Y_f$ such that $M(s,f) = X^s \cdot Y_f + \mu_f$ for some $\mu_f > 0$. Let $\sqrt{\cdot}$ denote the unique PSD square root. We define a set of functions $\cH$ by $h_{ij}(s) = (\sqrt{X^s})_{ij}$. Since $h_{ij} = h_{ji}$ there are only $\binom{k_1}{2}$ functions in $\cH$. We have
\begin{align*}
C(f) - f(s) &= X^s \cdot Y_f + \mu_f \\
&= \Tr[\sqrt{X^s}\sqrt{X^s}\sqrt{Y_f}\sqrt{Y_f}] + \mu_f \\
&= \Tr[(\sqrt{X^s}\sqrt{Y_f})^T\sqrt{X^s}\sqrt{Y_f}] + \mu_f \\
&= \sum_{ij} \left(\sum_k (\sqrt{X^s})_{ik}(\sqrt{Y_f})_{kj}\right)^2 + \mu_f \\
&= \sum_{ij} \left(\sum_k (\sqrt{Y_f})_{kj}h_{ik}(s)\right)^2 + \mu_f.
\end{align*}
Lastly, $\sigma h_{ij}(s) = h_{ij}(\sigma^{-1}s) = \sqrt{X^{\sigma^{-1}s}}_{ij} = \sqrt{\sigma^{-1}X^s}_{ij}$. Because $\sigma^{-1}$ is a coordinate permutation, its action on $X^s$ can be written $\sigma^{-1} X^s = P(\sigma) X^s P(\sigma)^T$. Then since $\left(P(\sigma)\sqrt{X^s}P(\sigma)^T\right)^2 = P(\sigma)X^sP(\sigma)^T = \sigma^{-1}X^s$, and the PSD square root is unique, we have $\sqrt{\sigma^{-1}X^s} = \sigma^{-1}\sqrt{X^s}$. Thus $h_{ij}(\sigma^{-1}s) = \sigma^{-1}\sqrt{X^s}_{ij} = \sqrt{X^s}_{\sigma^{-1}i\sigma^{-1}j} = h_{\sigma^{-1}i\sigma^{-1}j}(s)$, so indeed $\cH$ is $G$-invariant. 
\end{proof}

The second lemma is an old group-theoretic result. It has been used frequently in the context of symmetric LP and SDP formulations, see for example \cite{}. 
\begin{lemma}[(\cite{DIXANDMORT}, Theorems 5.2A and 5.2B)]\label{lem:dixandmort}
Let $n \geq 10$ and let $G \leq S_n$. If $|S_n: G| < \binom{n}{k}$ for some $k < n/4$, then there is a subset $I \subseteq [n]$ such that $|I| < k$, and $A([n] \setminus I)$ is a subgroup of $G$.
\end{lemma}
We are ready to prove the main theorem.
\begin{proof}[Proof of \prettyref{thm:symmetric-main}]
We start with the family of $\binom{r+1}{2} < \binom{n}{k_1}$ functions $\cH$ with the properties specified in \prettyref{lem:sdptosos}. We abuse notation slightly and just continue to write $\cH$ for the family of functions whose domain is $V(\cP)$ instead of $\cS$. There is no real difference since they are in bijection via $\phi$. For $h \in \cH$, obviously $|\text{Orb}(h)| \leq |\cH| < \binom{n}{k_1}$. By the orbit-stabilizer theorem, $|S_m : \text{Stab}(h)| = |\text{Orb}(h)| < \binom{n}{d}$, so by \prettyref{lem:dixandmort}, there is a $I \subseteq [m]$ of size at most $k_1$ such that $A([m] \setminus I) \leq \text{Stab}(h)$. Applying \prettyref{lem:functopoly}, we obtain polynomials $h'(x)$ of degree at most $k_2$ which agree with $h$ on $V(\cP)$. Then for each $f$ satisfying $\max f \leq S(f)$,
\[C(f) - o^f(\phi(s)) = \sum_i \left(\sum_{h \in \cH} \alpha_h \cdot h'_j(\phi(s))\right)^2 + \mu_f\]
for every $s \in \cS$. This is an equality on every point of $V(\cP)$ and each $h'$ is degree at most $k_2$, so $C(f) - o^f(x)$ is $2k_2$-SOS modulo $\gen{\cP}$. Thus the $k_2$th Theta Body relaxation is a $(C,S)$-approximate SDP relaxation of $\cM$. 
\end{proof}

\prettyref{thm:symmetric-main} and \prettyref{prop:tb_to_sos} immediately imply the following corollary:
\begin{corollary}\label{cor:lass-optimal}
Let $\cM = (\cS, \cF)$ have a $S_m$-symmetric $(k_1,k_2)$-block transitive boolean polynomial formulation $(\cP, \cO, \phi)$ on $n$ variables. If $\cP$ is $\ell$-effective, then if $\cM$ has any $(C,S)$-approximate, $S_m$-symmetric SDP relaxation of size $r < \sqrt{\binom{n}{k_1}}$ the $\ell k_2$th Lasserre relaxation is a $(C,S)$-approximate relaxation as well.
\end{corollary}
We also have collected several examples of combinatorial problems that we can apply \prettyref{cor:lass-optimal} to:
\begin{corollary}\label{cor:matching_bcsp_optimal}
\leavevmode
\begin{itemize}
\item If \textsc{Matching} has a $S_m$-symmetric SDP relaxation of size $r < \sqrt{\binom{n}{k}}$ achieving $(C,S)$-approximation, the $2k^2$ Lasserre relaxation is a $(C,S)$-approximate relaxation as well.
\item If \textsc{Balanced CSP} has a $S_m$-symmetric SDP relaxation of size $r < \sqrt{\binom{n}{k}}$ achieving $(C,S)$-approximation, the $k$th Lasserre relaxation is a $(C,S)$-approximate relaxation as well.
\end{itemize}
\end{corollary}
\begin{proof}
Follows immediately from \prettyref{ex:matching-blocktransitive}, \prettyref{ex:bcsp-blocktransitive}, \prettyref{thm:matching-effective}, \prettyref{thm:bcsp-effective}, and \prettyref{cor:lass-optimal}.
\end{proof}

\section{Optimality for \textsc{TSP}}
In this section we will prove that the Lasserre hierarchy is optimal for \textsc{TSP}. 
Recall that \textsc{TSP} has the following polynomial formulation on $n^2$ variables:
\begin{align*}
&\cP_n = \tsp \\
&\cO_n = \{\sum_{ij} d_{ij}\left(\sum_{k} x_{ik}x_{j,k+1}\right) | d_{ij} \in \R^+\} \\
&\phi_n(\tau) = \chi_{\tau}.
\end{align*}
This polynomial formulation is $S_m$-symmetric by the action $\tau(x_{ij}) = x_{\tau(i)j}$. Unfortunately, \textsc{TSP} is not block-transitive, so we are unable to exactly apply the framework of the previous section. However, it is very nearly block-transitive: 
\begin{lemma}\label{lem:tsp-blocktransitive}
If $I \subseteq [m]$, then let $J = I \times [m]$. Then $A([m] \setminus I)$ acts transitively on the elements of $V(\cP)$ that agree on $J$ and have the same sign.
\end{lemma}
\begin{proof}
If $\tau,\tau'$ are tours with $\phi(\tau)|_J = \phi(\tau')|_J$ and $\text{sign}(\tau) = \text{sign}(\tau')$, then let $\sigma = \tau'\tau^{-1}$. Clearly $\sigma\tau = \tau'$. For $i \in I$, $\sigma(i) = \tau'(\tau^{-1}(i)) = i$ since $\tau'$ and $\tau$ agree on $I$, thus $\sigma \in S([m]\setminus I)$. Because both $\tau'$ and $\tau$ have the same sign, $\sigma$ must be even, so $\sigma \in A([m]\setminus I)$.
\end{proof}

If we naively attempt the same strategy as in the previous section, we will run into the problem that the functions $\cH$ depend on the sign of the tour, which is a high degree function in this polynomial formulation. To handle the dependence on the sign of a tour, we need to embed any tour as an even tour on more vertices, then find a good approximation for \textsc{TSP} on the set of larger vertices. To this end, define the function $T: S_n \rightarrow A_{2n}$ by $T(\tau) = \tau \tau'$, where $\tau'$ fixes $[m]$, and $\tau'(i) = \tau(i)+m$ for $i \in \{m+1,\dots,2m\}$. Since $\sign(\tau) = \sign(\tau')$, $T(\tau)$ is indeed an even permutation. Also note that $T(\tau)|_{[m]}$ is a permutation of $[m]$, and indeed equal to $\tau$. Now we are ready to prove our main theorem:
\begin{theorem}\label{thm:tsp-opt}
If \textsc{TSP} on $2m$ vertices has an $A_{2m}$-coordinate symmetric SDP relaxation of size $r < \sqrt{\binom{n}{k}}$ achieving a $(C,S)$-approximation, then the $2k$th Lasserre relaxation is a $(C,S)$-approximate relaxation on \textsc{TSP} on $m$ vertices.
\end{theorem}
\begin{proof}
Let $f$ be an objective function for \textsc{TSP} on $m$ vertices, and let $F$ be the objective function for \textsc{TSP} on $2m$ vertices which has 
\[d_F(i,j+m) = d_F(i+m, j) = d_F(i+m, j+m) = d_F(i,j) = d_f(i,j).\] 
Then $F(T(\tau)) = 2f(\tau)$. Furthermore, if $\Pi \in S_{2m}$, then there exist tours $\tau_1$ of $[m]$ and $\tau_2$ of $\{m+1,\dots,2m\}$ such that $F(\Pi) = F(\tau_1\tau_2)$. This can be seen just by setting $\tau_1(i) = \Pi(i)$ or $\Pi(i) - m$, whichever is in $[m]$, and $\tau_2(i) = \Pi(i)$ or $\Pi(i)+m$, whichever is in $\{m+1,\dots,2m\}$. Clearly from the definition of $F$ this does not change the value. This implies that $\min_\Pi F(\Pi) = 2\min_{\tau} f(\tau)$.

Now if \textsc{TSP} has a symmetric SDP relaxation as in the theorem statement, by starting identically to \prettyref{thm:symmetric-main}, we obtain a family of $\binom{r+1}{2}$ functions $\cH$ which are $A_{2m}$-invariant and
\[F(\Pi) - \rho\min_{\Pi} F(\Pi) = \sum_i g_i^2(\Pi)\]
where each $g_i \in \gen{\cH}$. Furthermore, each $h \in \cH$ has a subset $I_h$ such that $h$ is stabilized by $A([2m] \setminus I_h)$ and $|I_h| \leq k$. Then by \prettyref{lem:tsp-blocktransitive}, the function $h$ depends only on the variables in $I_h \times [2m]$ and the sign of the permutation. The restriction of $h$ to the image of $T$ must then depend only on the variables in $I_h \times [2m]$, since every image of $T$ is an even permutation. Thus there exists a polynomial $h'(x)$ which depends only on the variables in $I_h \times [2m]$ which agrees with $h$ on the image of $T$. Because the polynomial formulation for \textsc{TSP} is boolean and we can eliminate monomials of the form $x_{ij}x_{i\ell}$ for $j \neq \ell$, the polynomial $h'(x)$ can be taken to have degree at most $|I| \leq k$. Finally, we note that $x_{ij} = x_{i+m,j+m}$ and $x_{i,j+m} = x_{i+m,j} = 0$ for every $i,j \in [m]$ on the image of $T$. Thus we can replace each instance of $x_{i+m,j+m}$ in $h'(x)$ with $x_{ij}$, and each instance of $x_{i,j+m}$ or $x_{i+m,j}$ with $0$ and not change the value of $h'(x)$ on the image of $T$. Now $h'$ depends only on variables $[m] \times [m]$, and since $T(\tau)$ restricted to these variables is $\tau$, we have the following polynomial identity:
\[F(T(\tau)) - \rho\min_{\Pi} F(\Pi) = \sum_i \left(\sum_{h \in \cH} \alpha_{ih} h'(\tau)\right)^2.\]
Now the LHS is equal to $2f(\tau) - 2\rho \min_{\tau} f(\tau)$, and thus $o^f(x) - \rho \min_{\tau} f(\tau)$ is $2k$-SOS modulo $\gen{\cP}$. Since this is true for every objective $f$, this implies that the $k$th Theta Body on $m$ vertices is a $\rho$-approximate SDP relaxation. Finally, by \prettyref{thm:tsp-effective}, we know that $\cP$ is $2$-effective, so the $2k$th Lasserre relaxation is also a $\rho$-approximate SDP relaxation.
\end{proof}

\section{Lower Bounds for \textsc{Matching}}
Recall that matching has a polynomial formulation on $\binom{n}{2}$ variables:
\begin{align*}
&\cP_n = \matching \\
&\cO_n = \{\sum_{(i,j) \in E} x_{ij} | E \subseteq [\binom{n}{2}]\} \\
&\phi_n(M) = \chi_M.
\end{align*}
By \prettyref{cor:matching_bsp_optimal}, the Lasserre hierarchy provides the best approximation for matching among small SDPs. However, it is also known that the Lasserre hierarchy, which certifies nonnegativity via PC$_>$ proofs, does not perform well on matching. In particular, it is incapable of certifying that the number of edges in the matching of an $n$-clique with $n$ odd is at most $(n-1)/2$ until $\Omega(n)$ rounds:
\begin{theorem}[Due to \cite{grigoriev}]\label{thm:grigoriev}
If $n$ is odd, $V(\cP_n) = \emptyset$, but every PC$_>$ refutation of $\cP_n$ has degree $\Omega(n)$.
\end{theorem}
Since Lasserre does poorly on matchings, we should be able to prove that every small symmetric SDP formulation must do poorly: 
\begin{theorem}
Let \textsc{Matching} have an $S_n$-coordinate-symmetric SDP relaxation of size $d$ that achieves a $(C,S)$-approximation with $C(f) = \max f + \epsilon/2$ and $S(f) = \max f$ for some $0 \leq \epsilon < 1$. Then $d \geq 2^{\Omega(n)}$.
\end{theorem}
\begin{proof}
Let $k$ be the smallest integer such that $d < \sqrt{\binom{n}{k}}$. Taking \prettyref{ex:matching-blocktransitive}, \prettyref{thm:matching-effective}, and \prettyref{cor:lass-optimal} together, the $2\binom{k}{2}$th Lasserre relaxation is a $(C,S)$-approximate SDP formulation for \textsc{Matching}. Actually if we are slightly more careful in our application of \prettyref{lem:functopoly}, we can show that the $k$th Lasserre relaxation suffices. For a set $I \subseteq [n]$, the associated subset of $\left[\binom{n}{2}\right]$ that satisfies the block-transitivity is $E(I,I)$, the set of edges lying entirely in $I$. This has size $\binom{k}{2}$, and so we can conclude that the polynomials $h'$ have degree at most $\binom{k}{2}$. However, by eliminating monomials containing $x_{ij}x_{i\ell}$ for $\ell \neq j$ (which are zero on $V(\cP)$), we can actually take the polynomials $h'$ to have degree at most $k/2$. 

Now let $m = n/2$ or $n/2-1$, whichever is odd. Let $A = [m]$, $B = \{m,\dots, 2m\}$, and if $m = n/2-1$, let $C = \{2m+1,2m+2\}$, otherwise $C = \emptyset$. Note that $A \cup B \cup C = [n]$ and they are all disjoint. Consider the objective function $f = f_{E(A,A)}$ and its associated polynomial $o^f(x) = \sum_{ij \in E(A,A)} x_{ij}$. Because the $k$th Lasserre relaxation achieves a $(\max f + \epsilon/2,\max f)$-approximation, and every $f$ satisfies the soundness condition, we know
\begin{align*}
C(f) - o^f(x) &= \frac{m-1}{2} + \frac{\epsilon}{2} - \sum_{ij \in E(A,A)} x_{ij} \\
&= \frac{1}{2}\sum_{\substack{i \in A \\ j \in B,C}} x_{ij} - \frac{1-\epsilon}{2}
\end{align*}
has a PC$_>$ proof of nonnegativity from $\cP_n$ of degree at most $2k$. Now we make a substitution in the polynomial identity: replace each instance of $x_{ij}$ with either $x_{i-m, j-m}$ if $i,j \in B$, or $0$ if $(i,j) \in (A,B), (A,C), (B,C)$. If $(i,j) \in (C,C)$, replace $x_{ij}$ with $1$. Note that under this substitution, every polynomial in $\cP_n$ is mapped to either $0$ or a polynomial in $\cP_m$. So if this substitution is made on a PC$_>$ proof from $\cP_n$, the result is a PC$_>$ proof from $\cP_m$, clearly of no higher degree. This implies that $-(1-\epsilon)/2$ has a degree-$2k$ PC$_>$ proof of nonnegativity from $\cP_m$. By \prettyref{thm:grigoriev}, this means that $k = \Omega(m)$, and since $m = \Theta(n)$, clearly $d \geq 2^{\Omega(n)}$.
\end{proof}
